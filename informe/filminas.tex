\documentclass[12pt,handout]{beamer}

%\documentclass[a4paper,12pt,spanish,oneside]{book}
\usepackage{amsmath}
\usepackage{amsthm}
\usepackage{amssymb}
\usepackage{enumerate}
\usepackage{verbatim}
\usepackage{makeidx}
% \usepackage{stmaryrd}

\usepackage[utf8]{inputenc}
\usepackage[spanish]{babel}
%\usepackage[latin1]{inputenc}
%\usepackage{t1enc}

\newlength{\ancho}
\setlength{\ancho}{\paperwidth}
\addtolength{\ancho}{-2in}
\textwidth=\ancho
%\topmargin=0pt
%\headsep=0pt
%\headheight=0pt
%\evensidemargin=0mm
%\oddsidemargin=0mm

%
%Caracteres

\newcommand{\Baire}{\mathcal N}
\newcommand{\Cantor}{\mathcal C}
\newcommand{\R}{\mathbb R}
\newcommand{\C}{\mathbb C}
\newcommand{\Q}{\mathbb Q}
\newcommand{\N}{\mathbb N}
\newcommand{\I}{\mathbb I}
\newcommand{\Qvar}{\mathcal Q}
\newcommand{\K}{\mathcal K}
\newcommand{\T}{\mathcal T}
\newcommand{\U}{\mathcal U}
\newcommand{\F}{\mathcal F}

%Varias
\newcommand{\subp}{\mathrm{M}}
\newcommand{\prob}{\mathrm{P}}
\newcommand{\graf}{\mathrm{graf}}
\newcommand{\diam}{\mathrm{diam}}
\newcommand{\B}{\mathbf{B}}
\newcommand{\longi}{\mathrm{long}}
\newcommand{\osc}{\mathrm{osc}}
\newcommand{\tends}{\rightarrow}

%
%	Estilo
\newcommand{\mc}[1]{\mathcal{#1}}
\newcommand{\mb}[1]{\mathbf{#1}}
\newcommand{\remark}[1]{\marginpar{\tiny{#1}}}
\renewcommand{\eqref}[1]{(\ref{#1})}
\newcommand{\teoref}[1]{teorema \ref{#1}}
\newcommand{\propref}[1]{proposicin \ref{#1}}
\newcommand{\lemaref}[1]{lema \ref{#1}}
\newcommand{\cororef}[1]{corolario \ref{#1}}

%
% 	Logicas
\newcommand{\true}{\mathsf{T}}
\newcommand{\sat}{\vDash}
\newcommand{\nsat}{\nvDash}
\newcommand{\sii}{\Leftrightarrow}
\newcommand{\logeq}{\Leftrightarrow}
\newcommand{\ent}{\rightarrow}
\newcommand{\tne}{\Leftarrow}
\newcommand{\iso}{\cong}		
\newcommand{\func}{\rightarrow}
\newcommand{\Func}{\longrightarrow}
\newcommand{\comp}{\circ}
\newcommand{\y}{\wedge}
\newcommand{\Y}{\bigwedge}
\renewcommand{\o}{\vee}
\newcommand{\abst}[1]{\llbracket #1\rrbracket}

% USO: \equ{i}{x} ---> p_i(x) = q_i(x)
\newcommand{\equ}[2]{p_{#1}(\vec{#2})=q_{#1}(\vec{#2})}

% USO: \quasiid{i}{x}
%\newcommand{\quasiid}[2]{\forall \vec{#2}(\Y_{#1=1}^n \equ{#1}{#2}) \ent \equ{}{#2}}
\newcommand{\quasiid}[2]{\forall \vec{#2}([(\Y_{#1=1}^n p_#1 = q_#1) \ent p = q]\vec{#2}}

% USO: \quasiid{n}
%\newcommand{\qi}[1]{\mathrm{qid}(p_1,q_1,\ldots,p_#1,q_#1,p,q)}
\newcommand{\qi}[1]{\mathrm{qid}(\{(p_i,q_i)\}_{i=1}^#1,p,q)}

%quasivariedad generada
\newcommand{\qgen}[1]{\mathrm{Q}(#1)}

%congruencia identidad
\newcommand{\id}[1]{\Delta^#1}

%
% Conjuntos
\newcommand{\union}{\ensuremath{\cup}}
\newcommand{\Union}{\ensuremath{\bigcup}}
\newcommand{\included}{\ensuremath{\subseteq}}
\newcommand{\includes}{\ensuremath{\supseteq}}
\newcommand{\notincluded}{\ensuremath{\not\subseteq}}
\newcommand{\inters}{\ensuremath{\cap}}
\newcommand{\Inters}{\ensuremath{\bigcap}}
\newcommand{\en}{\ensuremath{\in}}
\newcommand{\foral}{\ensuremath{\forall}}
\newcommand{\exi}{\ensuremath{\exists}}
\newcommand{\subi}[1]{\ensuremath{_{#1}}}
\newcommand{\compl}[1]{#1^c}
\newcommand{\f}[3]{{#1:#2\rightarrow#3}}
\newcommand{\vacio}{\emptyset}
\newcommand{\dom}{\mathrm{dom}}
\newcommand{\Pfin}{P^{<\omega}}

%
%	Griegas
\renewcommand{\th}{\theta}
\renewcommand{\a}{\alpha}
\renewcommand{\b}{\beta}
\newcommand{\g}{\gamma}
\newcommand{\Ga}{\Gamma}
\renewcommand{\d}{\delta}
\newcommand{\eps}{\epsilon}
\newcommand{\e}{\varepsilon}
\newcommand{\vp}{\varphi}
\newcommand{\sig}{\sigma}
\newcommand{\Sig}{\Sigma}
\renewcommand{\l}{\lambda}
\newcommand{\La}{\Lambda}
\newcommand{\om}{\omega}
\newcommand{\Om}{\Omega}
\newcommand{\z}{\zeta}

%
%	Entornos de redaccion
\newenvironment{dem}{\begin{proof}[Prueba]}{\end{proof}}
\newenvironment{ideadem}{\begin{proof}[Idea de prueba]}{\end{proof}}
\newenvironment{sol}{\begin{proof}[Solución]}{\end{proof}}
%
\theoremstyle{definition}
\newtheorem{defn}{Definición}
%
\theoremstyle{remark}
\newtheorem{ejem}{Ejemplo}
\newtheorem{nota}{Nota}
\newtheorem{observ}{Observación}
%
\theoremstyle{plain}
\newtheorem{lem}{Lema}
\newtheorem{af}[lem]{Afirmación}
\newtheorem{teo}[lem]{Teorema}
\newtheorem{coro}[lem]{Corolario}
\newtheorem{prop}[lem]{Proposición}


%%%%%%%%%%% Lo que viene  a continuacion es el tipo de presentacion que quieres, las presentaciones se llaman con nombres de ciudades puedes cambiarlas y tomar
%%%%%%%%la que mas te guste.
\mode<presentation> {
  %\usetheme{Frankfurt}
  %\usetheme{Warsaw}
  %\usetheme{Darmstadt}
  \usetheme{Dresden}
  %\usetheme{Singapore}
  %\usetheme{Bergen}
  %\usetheme{Boadilla}
  %\usetheme{Berkeley}
  \setbeamercovered{transparent}
  %\setbeamertemplate{background canvas}[vertical shading][bottom=red!20,top=yellow!30]
%%%%%%%%%%%%%%%%%%%%%%%%%%%%%%%%%%
%%%%%%% aqui vienen los colores

  %\usecolortheme{crane}
  %\usecolortheme{seahorse}
  \usecolortheme{whale}
  %\usecolortheme{rose}
  %\usecolortheme{orchid}
} \usepackage{alltt}

\newenvironment{stepenumerate}{\begin{enumerate}[<+->]}{\end{enumerate}}
\newenvironment{stepitemize}{\begin{itemize}[<+->]}{\end{itemize} }
\newenvironment{stepenumeratewithalert}{\begin{enumerate}[<+-| alert@+>]}{\end{enumerate}}
\newenvironment{stepitemizewithalert}{\begin{itemize}[<+-| alert@+>]}{\end{itemize} }
\newtheorem{propo}{Proposition}

%%%%%%%los paquetes. 
\usepackage{amssymb,amsmath,latexsym}
%\usepackage[mathcal]{euscript}
%\usepackage[polish]{babel}
\usepackage{color}
\usepackage{hyperref}
%\usepackage{dsfont}
%\usepackage[normalem]{ulem}
\usepackage{enumerate}
%\usepackage[all,2cell,dvips]{xy} \UseAllTwocells \SilentMatrices
%\usepackage[utf8]{inputenc}

\title{Análisis de la herramienta TLA+ Proof System}
%
%
\author{Pablo Celayes, Giovanni Rescia, Ariel Wolfmann}

%
\institute{Facultad de Matemática, Astronomía y Física\\
Universidad Nacional de Córdoba}

\date{Junio 2015}


\begin{document}
%%%%%%%%%%%%%%%%%%%%%%%%%%%PAGINA DEL TITULO
\begin{frame}
  \titlepage
\end{frame}

%%%%%%%%%%%%%%%%%  tODO LO QUE QUIERAS PONER EN LOS FRAMES.
% \begin{frame}
%   \frametitle{EFD-sentences}
%   \tableofcontents[pausesections]
% %  \tableofcontents[pausesections]
%   %You might wish to add the option [pausesections]
% \end{frame}

%%%%%%%%%%%%%%%%%%%%%%%% EFDs & Algebraic Functions %%%%%%%%%%%%%%%%%%%%%%%%%%%%
\section{EFDs \& Algebraic Functions}

\begin{frame}
\frametitle{EFD-sentences}

Let $\mathcal{L}$ be an algebraic first-order language. 
\pause

An \textit{Equational Function Definition Sentence}
(\textit{EFD for short}) is a sentence of the form
\pause
\[
\varphi =\forall \vec{x} \; \exists ! \vec{z}\;\bigwedge_{i=1}^{k}p_{i}(%
\vec{x},\vec{z})=q_{i}(\vec{x},\vec{z})\text{.} 
\] 
\pause
where $p_i$ and $q_i$ are $\mathcal{L}$-terms.

\end{frame}
\begin{frame}
\frametitle{Algebraic Functions}
Suppose $\varphi$ is an EFD, and $\mathbf{A} \sat \varphi$.
\bigskip

The \textit{function defined by $\varphi$} is the map
\bigskip

\[ [\varphi]^{\mathbf{A}}: \mathbf{A}^n \to \mathbf{A}^m \]

given by 

$ [\varphi]^{\mathbf{A}} (\vec{a})$ = unique $\vec{b}$ such that 

\[\bigwedge_{i=1}^{k}p_{i}(\vec{a},\vec{b})=q_{i}(\vec{a},\vec{b}) \]

\end{frame}

\begin{frame}

We denote by $[\varphi]^{\mathbf{A}}_i$ the coordinates of this map, ie, $[\varphi]^{\mathbf{A}}_i = \pi_i \circ [\varphi]^{\mathbf{A}}$

\bigskip

We use $E(\varphi)$ (resp. $U(\varphi)$) to denote the \textit{existence sentence of $\varphi$} (resp. \textit{unicity sentence of $\varphi$}):

\[
\begin{array}{rcl}
E(\varphi) & = & \forall \vec{x} \; \exists \vec{z}\;\bigwedge_{i=1}^{k}p_{i}(\vec{x},\vec{z})=q_{i}(\vec{x},\vec{z})\\
U(\varphi) & = & \forall \vec{x} \; \forall \vec{z} \; \forall \vec{w}\;
\bigwedge_{i=1}^{k}p_{i}(\vec{x},\vec{z})=q_{i}(\vec{x},\vec{z}) \\ 
& & \wedge (\bigwedge_{i=1}^{k}p_{i}(\vec{x},\vec{w})=q_{i}(\vec{x},\vec{w})) \rightarrow  (\vec{z} = \vec{w})\\
\end{array}
\]

\end{frame}

\begin{frame}

\begin{definition}

\bigskip

A sentence of the form $[\varphi]_{i}^\mathbf{A}$ is called an \textit{algebraic function over $\mathbf{A}$}

\pause

\bigskip

An \textit{algebraically expandable (AE) class} is a class $\mathcal{C}$ of algebras definable by EFDs, 

\bigskip
\pause

i.e.: $\mathcal{C}=Mod(\Sigma )$, for some set $\Sigma$ of EFDs  

\end{definition}

\end{frame}

\begin{frame}
 \begin{definition}
  We say that $\mathbf{B}$ is a \textit{daughter} of $\mathbf{A}$ 

(or $\mathbf{A}$ is a \textit{mother} of $\mathbf{B}$) if 
\[ \mathbf{B} \in H(\mathbf{A}) \cap IS(\mathbf{A}) \]

 \end{definition}

\textbf{Remark} 

If 

\begin{stepitemize}
 \item $\varphi$ is an EFD,
 \item $\mathbf{A} \sat \varphi$ and 
 \item $\mathbf{B}$ is a daughter of $\mathbf{A}$
\end{stepitemize}

\pause

Then

\[ \mathbf{B} \sat \varphi \]




\end{frame}


\section{Global Products}

\begin{frame}
 \frametitle{Global Products}
\begin{Definition}

\pause 
\bigskip

A subdirect product 

\pause

\[ \mathbf{A} \subseteq_{SD} \prod_{i \in I} \{\mathbf{A}_i : i \in I\} \]

\pause

is \textit{global} if there is a topology $\tau$ over $I$ such that

\begin{stepitemize}

 \item $\tau({E(x,y) : x,y \in \mathbf{A}}) \subseteq \tau$

\item If $x \in \prod_{i \in I} \{\mathbf{A}_i : i \in I\}$ and $E(x,y) \in \tau$ for all $y \in \mathbf{A}$, then $x \in \mathbf{A}$

\end{stepitemize}
\end{Definition}
\end{frame}

\begin{frame}

\begin{lemma}

\bigskip

 If $\varphi$ is an EFD, $\{A_i : i \in I\} \sat \varphi$ and $\mathbf{A}$ is a global product of the algebras 
$\{\mathbf{A}_i : i \in I\}$ then $\mathbf{A} \sat \varphi$

\end{lemma}
 
\end{frame}

\section{The case of $\mathcal{D}_{01ab}$}
\begin{frame}
 \frametitle{The case of $\mathcal{D}_{01ab}$}

\begin{lemma}
 Every lattice in $\mathcal{D}_{01ab}$ can be represented as a global subdirect product with factors in the subclass
$\mathcal{G}$ of all two- and three-element members
% \{\mathbf{2}_{00}, \mathbf{2}_{01}, \mathbf{2}_{10}, \mathbf{2}_{11}, \mathbf{3}_{00}, \mathbf{3}_{0M}, 
% \mathbf{3}_{01}, \mathbf{3}_{M0}, \mathbf{3}_{MM}, \mathbf{3}_{M1}, \mathbf{3}_{10}, \mathbf{3}_{1M}, \mathbf{3}_{11} \}
\end{lemma}
\end{frame}


\begin{frame}

\frametitle{Starting with Maximal Subclasses}

\[
\begin{array}{c}
Mod(\varphi) = \mathcal{G} - \{\mathbf{A}\} \\ 

\pause
\bigskip

Mod(\psi) = \mathcal{G} - \{\mathbf{B}\}
\end{array}
\]

\pause
\bigskip

\begin{center}
 $Mod(\varphi \wedge \psi) = \mathcal{G} - \{\mathbf{A},\mathbf{B}\}$
\end{center}
 
\end{frame}









\section{$\mathcal{G} - \{\mathbf{3}_{MM}\}$ is not AE}

\begin{frame}

\frametitle{$\mathcal{G} - \{\mathbf{3}_{MM}\}$ is not AE}
 
\begin{propo}
 If $\mathcal{C}$ is an AE subclass of $\mathcal{D}_{01ab}$ containing $\mathcal{G}' = \{\mathbf{2}_{00}, \mathbf{2}_{11}, 
\mathbf{3}_{0M}, \mathbf{3}_{M0}, \mathbf{3}_{M1}, \mathbf{3}_{1M}\}$ and any member $\mathbf{L}$ of $\mathcal{G}$ with 
$\{a^{\mathbf{L}},b^{\mathbf{L}}\} = \{0,1\}$, 
then $\mathbf{3}_{MM} \in \mathcal{C}$
\end{propo}

\end{frame}


\begin{frame}

\bigskip

If there existed an EFD $\varphi$ such that $\mathcal{G}' \sat \varphi$ and $\mathbf{3}_{MM} \nvDash \varphi$...

\pause

\bigskip

1) We can assume $\varphi = \exists ! z \; \bigwedge_{i=1}^{k}p_{i}(z) = q_{i}(z)$. 


\begin{stepitemize}

\item Use the \textit{translation property} that holds in $(\mathcal{D}_{01ab})_{SI}$:

\pause

$\mathbf{2}_{ab} \sat \forall x \; y \; (x = y) \vee (z = w) \Leftrightarrow$

$z|_{x \wedge y}^{x \vee y} =  w|_{x \wedge y}^{x \vee y}$

\pause
\bigskip

where $x|_u^v = (x  \vee u) \wedge v$

%$\mathbf{2}_{ab} \sat \forall x y (x = y) \vee (z = w) \Leftrightarrow$ 

%$((x \wedge y) \vee z) \wedge (x \vee y) =  ((x \wedge y) \vee z) \wedge (x \vee y)$


\end{stepitemize}

\end{frame}

\begin{frame}

2) Programming techniques...

\pause
\bigskip

... to see that $\varphi$ should be

\[ \exists ! z \; (a \wedge b \wedge z = 0) \wedge (a \vee b \vee z = 1) \wedge \varepsilon(z) \]

\bigskip

for some set of equations $\varepsilon(z)$

\end{frame}

\begin{frame}
 
3) Finally, more programming...

\pause
\bigskip

... to conclude that there is no possible choice for $\varepsilon(z)$.

\bigskip

\bigskip

\pause

But

\[ \exists ! z \; (a \wedge b \wedge z = 0) \wedge (a \vee b \vee z = 1) \]

\pause

holds for no $\mathbf{L}$ in $\mathcal{\mathcal{G}}$ with $\{a^{\mathbf{L}},b^{\mathbf{L}}\} = \{0,1\}$


\end{frame}






\section{Axiomatizations in $\mathcal{G}$}

\begin{frame}
 \frametitle{Maximal Classes Axiomatizations}

\begin{tabular}{ | l | l | }

\hline                       \

$\varphi$ & $Mod_{\mathcal{G}}(\varphi)$ \\

\hline

$\exists ! z \; (a \vee b) \wedge z = 0$     	& \pause  $\mathcal{G} - \{\mathbf3_{00} , \mathbf2_{00}\}$ %"a s b != 0"
\\

\pause

$\exists ! z \; (a \wedge b) \vee z = 1$     	& \pause $\mathcal{G} - \{\mathbf3_{11} , \mathbf2_{11}\}$ %"a i b != 1"
\\

\pause

$\forall x \; \exists ! z \; ( x|_{a \vee b}^1 \wedge z = a \vee b ) 

\wedge  (x|_{a \vee b}^1 \vee z = 1) $ & \pause $\mathcal{G} - \{\mathbf3_{00}\}$ %"[a s b, 1] complementado"
\\

\pause

$\forall x \; \exists ! z \; (x|_{a}^{a \vee b} \wedge z = a)

\wedge (x|_{a}^{a \vee b} \vee z = a \vee b)$ 
& \pause $\mathcal{G} - \{\mathbf3_{01}\}$ %"[a, a s b] complementado"
\\

\pause

$\forall x \; \exists ! z \; ( x|_0^{a \wedge b} \wedge z = 0) 

\wedge  (x|_0^{a \wedge b} \vee z = a \wedge b)$ & \pause $\mathcal{G} - \{\mathbf3_{11}\}$ %[0, a i b] complementado"
 \\

\pause

$ \forall x \; \exists ! z \; (x|_{b}^{a \vee b} \wedge z = b)

\wedge (x|_{b}^{a \vee b} \vee z = a \vee b)$ & \pause $\mathcal{G} - \{\mathbf3_{10}\}$ %	"[b, a s b] complementado"
 \\

\pause

$\exists ! z \; ( a|_b^1 \wedge z = b ) 

\wedge  (a|_b^1 \vee z = 1)$ & \pause $\mathcal{G} - \{\mathbf3_{M0}\}$ %" a complementado en [b,1]" 
\\

\pause

$\exists ! z \; ( a|_0^b \wedge z = 0 ) 

\wedge  (a|_0^b \vee z = b)$ & \pause $\mathcal{G} - \{\mathbf3_{M1}\}$ %" a complementado en [0,b]" 
\\

\pause

$\exists ! z \; ( b|_a^1 \wedge z = a ) 

\wedge  (b|_a^1 \vee z = 1)$ & \pause $\mathcal{G} - \{\mathbf3_{0M}\}$ %" b complementado en [a,1]" 
\\

\pause

$\exists ! z \; ( b|_0^a \wedge z = 0 ) 

\wedge  (b|_0^a \vee z = a)$ & \pause $\mathcal{G} - \{\mathbf3_{1M}\}$ % b complementado en [0,a]" 
\\

\hline
\end{tabular}

\bigskip

\end{frame}

\begin{frame}
 
\begin{propo}
 
Suppose $\mathcal{C}$ is a subclass of $\mathcal{G}$ closed under the relationship of daughtership that contains 
\[ \{\mathbf{2}_{01}, \mathbf{2}_{10}, \mathbf{3}_{MM} \} \]
Then there is an EFD $\varphi$ such that 
\[ \mathcal{C} = Mod(\varphi) \cap \mathcal{G} \]
\end{propo}

\end{frame}


\begin{frame}%%%%%%%%%%
\frametitle{References}

\begin{thebibliography}{CamVag2009}

%\pause
\bibitem[KrCla79]{AE-classes}\textrm{Krauss, P.H., Clark, D.M.}, 
\textit{Global Subdirect Products}, Amer. Math. Soc. Mem. \textbf{210} (1979).

%\pause
\bibitem[GraVag96]{AE-classes}\textrm{H. Gramaglia, D. Vaggione}, 
\textit{Birkhoff-like Sheaf Representations of Varieties of Lattices Expansions}, Studia Logica \textbf{56} (1996), 111-131.

%\pause
\bibitem[CamVag09]{AE-classes}\textrm{M. Campercholi, D. Vaggione}, \textit{Algebraically Expandable
Classes}, Algebra Universalis \textbf{61} (2009), no. 2, 151-186.

%\pause
\bibitem[CamVag10]{Afunc}\textrm{M. Campercholi, D. Vaggione}, \textit{Algebraic Functions}, Studia Logica \textbf{98}, 285-306 (2011).



\end{thebibliography}

\end{frame}

\begin{frame}
\begin{figure}[center]
% \includegraphics[width=95mm]{19127354.jpg}
%\caption{algebraically closed clones in Post's Lattice} 
\end{figure}
\end{frame}

\end{document}
