% This is LLNCS.DEM the demonstration file of
% the LaTeX macro package from Springer-Verlag
% for Lecture Notes in Computer Science,
% version 2.4 for LaTeX2e as of 16. April 2010
%
\documentclass[spanish]{llncs}
%
\usepackage[spanish]{babel}
\usepackage[utf8]{inputenc}
\usepackage{makeidx}  % allows for indexgeneration
%
\begin{document}
%

\mainmatter              % start of the contributions
%
\title{Análisis de la herramienta TLA+ Proof System}
%
%
\author{Pablo Celayes \and Giovanni Rescia \and Ariel Wolfmann}

%
\institute{Facultad de Matemática, Astronomía y Física\\
Universidad Nacional de Córdoba}

\maketitle              % typeset the title of the contribution

\begin{abstract}
Agente: Cuando yo le diga hola señor Thompson, usted dice, hola

Homero: ¡Bien!

Agente: ¡¡Hola señor Thompson!!

Homero: ... ...

Agente: ¡¡Recuerde!! ¡su nombre ahora es Homero Thompson!

Homero: ¡¡Enterado!!

Agente: ¡¡Hola señor Thompson!!

Homero: (al otro agente)¡¡Creo que le habla a usted!!!. \dots

\keywords{model checking, proof system, recibirse}
\end{abstract}
%

\section{Contexto de creación de la herramienta}

 En 1977, Amir Pnueli introdujo el uso de lógica temporal para describir el comportamiento de un sistema. En principio, un sistema se podía describir con una sola fórmula de lógica temporal. En la práctica no era así. La lógica temporal de Pnueli era ideal para describir ciertas propiedades de los sistemas, pero poco adecuada para muchas otras. Es por ello que normalmente se la combinaba con maneras más tradicionales de describir sistemas. \cite{pnueli}

A fines de los 80's Leslie Lamport inventó el lenguaje TLA (Temporal Logic of Actions,
una variante simple de la lógica temporal de Pnueli). TLA facilita la tarea de describir un sistema completa en una sola fórmula. La mayor parte de una especificación TLA se compone de Matemática ordinaria, no temporal. La lógica temporal sólo juega un rol significativo en la descripción de aquellas propiedades para las que realmente es buena. \cite{specsys}

En 2001, Lamport comenzó a trabajar en el centro \textit{Microsoft Research} de Mountain View, Estados Unidos. De esta etapa surge el proyecto TLA+ Proof Systema (TLAPS) que actualmente se desarrolla como parte del proyecto \textit{Tools for Proofs} en el centro conjunto de investigación \textit{Microsoft Research - INRIA} de Palaiseau, Francia. El proyecto cuenta con una activa lista de mail para discusiones relativas a su uso y un sistema público de seguimiento de \textit{bugs}.


\section{Objetivo de la herramienta}

TLAPS es una herramienta que verifica mecánicamente la correctitud
de pruebas escritas en TLA+.
TLA+ es un lenguaje de especificación de propósito general, orientado a sistemas
concurrentes y distribuidos.

En general, una prueba de TLA+, es una colección de sentencias estructuradas jerárquicamente,
donde cada sentencia tiene una afirmación, injustificada o justificada por una colección de hechos citados.

El propósito de TLAPS es verificar las pruebas de teoremas propuestas por el usuario, es decir,
que la jerarquía de las sentencias de hecho establecen la veracidad del teorema si las afirmaciones fueran ciertas,
y luego verificar que la afirmación de cada sentencia justificada es implicada por los hechos citados.

Si un teorema de TLA+ tiene una prueba con todas sus afirmaciones justificadas, entonces, como resultado
de comprobar la prueba, TLAPS verifica si el teorema es cierto.

\section{Descripción de la herramienta del lado del usuario}

TLA+ es una herramienta de alto nivel para la descripción de los sistemas, especialmente sistemas concurrentes asíncronos y distribuidos. Fue diseñado para ser simple, muy expresivo, 
y permitir una formalización directa del razonamiento de aserciones tradicional.
Se fundamenta en la idea de que la mejor manera de describir formalmente es con matemática simple, y que un lenguaje de especificación debe contener lo menos posible más allá de lo 
que se necesita para escribir matemática simple con precisión.

Para cumplir con el objetivo de formalizar el razonamiento de aserciones, TLA+  esta basado en TLA (Temporal Logic of Action), una variante simple de la lógica temporal lineal.

TLA+ utiliza a PlusCal como lenguaje de especificación, para el modelado se basa en el modelo Standard, TLC es quien se encarga de verificar la corrección de los modelos y 
finalmente TLA+ Proof System es un asistente de pruebas, para verificar las especificaciones dadas en TLA+, 
en esta herramienta específica es donde nos enfocaremos el resto del documento. 

Cabe aclarar que TLA+ posee una interfaz gráfica donde el usuario puede utilizar todas estas herramientas de manera sencilla e integrada.

  \subsection{PlusCal}
  TLA+ utiliza PlusCal como lenguaje en el cual se debe realizar la especificación.
  PlusCal es un lenguaje algorítmico que, a primera vista, parece un lenguaje de programación imperativo pequeño. Sin embargo, una expresión PlusCal puede ser cualquier expresión de 
  TLA+, lo que significa cualquier cosa que puede ser expresado con matemática. Esto hace a PlusCal mucho más expresivo que cualquier lenguaje de programación.
  Un algoritmo PlusCal se traduce (compila) en una especificación TLA+, la cual puede ser chequeada con las herramientas de TLA+.
  Fue diseñado para  reemplazar al  pseudocódigo, conservando su sencillez y al mismo tiempo proporciona un lenguaje formalmente definido y verificable, en el cual las personas 
  sin tantos conocimientos profundos en matematica pueden definir su especificacion, de forma sencilla y facil de entender y luego la herramienta se encarga de traducir esta a TLA+.

  \subsection{Modelo STANDARD}
  En un model Standard, un sistema abstracto se describe como un conjunto de comportamientos, cada uno representando una posible ejecución del sistema, donde un comportamiento es 
  una secuencia de estados y un estado es una asignación de valores a las variables.
  En este modelo, un evento, también llamado un paso, es la transición de un estado a otro en un comportamiento.

  \subsection{TLC}
  El corrector de modelos TLC construye un modelo de estado finito de las especificaciones TLA+  para el control de las propiedades de invariancia. Genera un conjunto de estados 
  iniciales que satisfacen la especificación, a continuación, realiza una búsqueda en amplitud (Breadth-First Search) sobre todas las transiciones de estado definidos. 
  La ejecución se detiene cuando todas las transiciones de estado conducen a estados que ya han sido descubiertos. Si TLC descubre un estado que viola una invariante del sistema, 
  se detiene y ofrece una traza infractora. En caso de que haya alcanzado un estado que no tenga posibles acciones habilitadas, 
  reporta un mensaje de error explicitando el posible deadlock(opcional).
  
  TLC ofrece un método de declarar simetrías modelo para defenderse contra explosión combinatoria. También paraleliza el paso de la exploración del estado, 
  y se puede ejecutar en modo distribuido para distribuir la carga de trabajo a través de un gran número de computadoras.
  
  TLA+ es un ejemplo de lenguaje de mucha expresividad, puede ser fácilmente
  utilizado para especificar un programa que acepte una máquina de Turing arbitraria como entrada y puede determinar si se detendrá o no. 
  Ningún corrector de modelos puede manejar todas las especificaciones TLA+. TLC maneja un subconjunto de TLA+ que intenta incluir la mayoría de las especificaciones algorítmicas 
  y propiedades de corrección, así como todas las especificaciones de diseño de protocolos y sistemas.

  \subsection{TLA+ Proof System}
  TLAPS, el sistema de prueba de TLA+, es una plataforma que se extiende de TLA+, para el desarrollo y verificación mecánica de demostraciones. 
  
  El lenguaje de pruebas de TLA+ es declarativo y requiere cierto conocimiento previo de matematica elemental. Soporta desarrollo incremental y  verifica la estructura jerárquica 
  de la demostracion.
  
  La herramienta traduce un demostración en un conjunto de pruebas independientes, y llama a una colección de verificadores que se encargan de chequearlas, 
  que incluyen demostradores de teoremas, asistentes de pruebas, chequeadores de satisfabilidad y procedimientos de decisión.
  
  La versión actualmente disponible TLAPS maneja casi toda la parte no temporal de TLA+, como así también el razonamiento temporal necesario para probar propiedades de 
  seguridad (safety) estándares, en particular, invariantes y simulación de transiciones, pero no para propiedades de vitalidad (liveness).


\section{Aspectos técnicos de la herramienta}
\section{Casos de estudio (exitosos o no) de la herramienta}
\section{Comparación con otras herramientas}
\section{Caso de estudio elegido}
\section{Conclusiones particulares}


%
% ---- Bibliography ----
%
\begin{thebibliography}{5}
%
\bibitem {pnueli}
Pnueli, Amir

The temporal logic of programs

Proceedings of the 18th IEEE Symposium on Foundation of Computer Science, 1977

\bibitem {specsys}
Lamport, Leslie

Specifying Systems: The TLA+ Language and Tools for Hardware and Software Engineers

Addison-Wesley. ISBN 0-321-14306-X, 2002




\end{thebibliography}


\clearpage
\addtocmark[2]{Author Index} % additional numbered TOC entry
\renewcommand{\indexname}{Author Index}
\printindex
\clearpage
\addtocmark[2]{Subject Index} % additional numbered TOC entry
\markboth{Subject Index}{Subject Index}
\renewcommand{\indexname}{Subject Index}
\input{subjidx.ind}
\end{document}
