% This is LLNCS.DEM the demonstration file of
% the LaTeX macro package from Springer-Verlag
% for Lecture Notes in Computer Science,
% version 2.4 for LaTeX2e as of 16. April 2010
%
\documentclass[spanish]{llncs}
%
\usepackage[spanish]{babel}
\usepackage[utf8]{inputenc}
\usepackage{makeidx}  % allows for indexgeneration
%
\begin{document}
%

\mainmatter              % start of the contributions
%
\title{Análisis de la herramienta TLA+ Proof System}
%
%
\author{Pablo Celayes \and Giovanni Rescia \and Ariel Wolfmann}

%
\institute{Facultad de Matemática, Astronomía y Física\\
Universidad Nacional de Córdoba}

\maketitle              % typeset the title of the contribution

\begin{abstract}
Agente: Cuando yo le diga hola señor Thompson, usted dice, hola

Homero: ¡Bien!

Agente: ¡¡Hola señor Thompson!!

Homero: ... ...

Agente: ¡¡Recuerde!! ¡su nombre ahora es Homero Thompson!

Homero: ¡¡Enterado!!

Agente: ¡¡Hola señor Thompson!!

Homero: (al otro agente)¡¡Creo que le habla a usted!!!. \dots

\keywords{model checking, proof system, recibirse}
\end{abstract}
%

\section{Contexto de creación de la herramienta}

Pablo

\section{Objetivo de la herramienta}

TLAPS es una herramienta que verifica mecánicamente la correctitud
de pruebas escritas en TLA+.
TLA+ es un lenguaje de especificación de propósito general, orientado a sistemas
concurrentes y distribuidos.

En general, una prueba de TLA+, es una colección de sentencias estructuradas jerárquicamente,
donde cada sentencia tiene una afirmación, injustificada o justificada por una colección de hechos citados.

El propósito de TLAPS is verificar las pruebas de teoremas propuestas por el usurio, es decir,
que la jerarquía de las sentencias de hecho establecen la veracidad del teorema si las afirmaciones fueran ciertas,
y luego verificar que la afirmación de cada sentencia justificada es implicada por los hechos citados.

Si un teorema de TLA+ tiene una prueba con todas sus afirmaciones justificadas, entonces, como resultado
de comprobar la prueba, TLAPS verifica si el teorema es cierto.

\section{Descripción de la herramienta del lado del usuario}

Oso (pedí ayuda si es mucho!)

\section{Aspectos técnicos de la herramienta}
\section{Casos de estudio (exitosos o no) de la herramienta}
\section{Comparación con otras herramientas}
\section{Caso de estudio elegido}
\section{Conclusiones particulares}


%
% ---- Bibliography ----
%
\begin{thebibliography}{5}
%
\bibitem {clar:eke}
Clarke, F., Ekeland, I.:
Nonlinear oscillations and
boundary-value problems for Hamiltonian systems.
Arch. Rat. Mech. Anal. 78, 315--333 (1982)

\bibitem {clar:eke:2}
Clarke, F., Ekeland, I.:
Solutions p\'{e}riodiques, du
p\'{e}riode donn\'{e}e, des \'{e}quations hamiltoniennes.
Note CRAS Paris 287, 1013--1015 (1978)

\bibitem {mich:tar}
Michalek, R., Tarantello, G.:
Subharmonic solutions with prescribed minimal
period for nonautonomous Hamiltonian systems.
J. Diff. Eq. 72, 28--55 (1988)

\bibitem {tar}
Tarantello, G.:
Subharmonic solutions for Hamiltonian
systems via a $\bbbz_{p}$ pseudoindex theory.
Annali di Matematica Pura (to appear)

\bibitem {rab}
Rabinowitz, P.:
On subharmonic solutions of a Hamiltonian system.
Comm. Pure Appl. Math. 33, 609--633 (1980)

\end{thebibliography}


\clearpage
\addtocmark[2]{Author Index} % additional numbered TOC entry
\renewcommand{\indexname}{Author Index}
\printindex
\clearpage
\addtocmark[2]{Subject Index} % additional numbered TOC entry
\markboth{Subject Index}{Subject Index}
\renewcommand{\indexname}{Subject Index}
\input{subjidx.ind}
\end{document}
