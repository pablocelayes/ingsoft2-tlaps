% This is LLNCS.DEM the demonstration file of
% the LaTeX macro package from Springer-Verlag
% for Lecture Notes in Computer Science,
% version 2.4 for LaTeX2e as of 16. April 2010
%
\documentclass[spanish]{llncs}
%
\usepackage[spanish]{babel}
\usepackage[utf8]{inputenc}
\usepackage{makeidx}  % allows for indexgeneration
%
\begin{document}
%

\mainmatter              % start of the contributions
%
\title{Análisis de la herramienta TLA+ Proof System}
%
%
\author{Pablo Celayes \and Giovanni Rescia \and Ariel Wolfmann}

%
\institute{Facultad de Matemática, Astronomía y Física\\
Universidad Nacional de Córdoba}

\maketitle              % typeset the title of the contribution

\begin{abstract}
Agente: Cuando yo le diga hola señor Thompson, usted dice, hola

Homero: ¡Bien!

Agente: ¡¡Hola señor Thompson!!

Homero: ... ...

Agente: ¡¡Recuerde!! ¡su nombre ahora es Homero Thompson!

Homero: ¡¡Enterado!!

Agente: ¡¡Hola señor Thompson!!

Homero: (al otro agente)¡¡Creo que le habla a usted!!!. \dots

\keywords{model checking, proof system, recibirse}
\end{abstract}
%

\section{Contexto de creación de la herramienta}

 En 1977, Amir Pnueli introdujo el uso de lógica temporal para describir el comportamiento de un sistema. En principio, un sistema se podía describir con una sola fórmula de lógica temporal. En la práctica no era así. La lógica temporal de Pnueli era ideal para describir ciertas propiedades de los sistemas, pero poco adecuada para muchas otras. Es por ello que normalmente se la combinaba con maneras más tradicionales de describir sistemas. \cite{pnueli}

A fines de los 80's Leslie Lamport inventó el lenguaje TLA (Temporal Logic of Actions,
una variante simple de la lógica temporal de Pnueli). TLA facilita la tarea de describir un sistema completa en una sola fórmula. La mayor parte de una especificación TLA se compone de Matemática ordinaria, no temporal. La lógica temporal sólo juega un rol significativo en la descripción de aquellas propiedades para las que realmente es buena. \cite{specsys}

En 2001, Lamport comenzó a trabajar en el centro \textit{Microsoft Research} de Mountain View, Estados Unidos. De esta etapa surge el proyecto TLA+ Proof Systema (TLAPS) que actualmente se desarrolla como parte del proyecto \textit{Tools for Proofs} en el centro conjunto de investigación \textit{Microsoft Research - INRIA} de Palaiseau, Francia. El proyecto cuenta con una activa lista de mail para discusiones relativas a su uso y un sistema público de seguimiento de \textit{bugs}.


\section{Objetivo de la herramienta}

TLAPS es una herramienta que verifica mecánicamente la correctitud
de pruebas escritas en TLA+.
TLA+ es un lenguaje de especificación de propósito general, orientado a sistemas
concurrentes y distribuidos.

En general, una prueba de TLA+, es una colección de sentencias estructuradas jerárquicamente,
donde cada sentencia tiene una afirmación, injustificada o justificada por una colección de hechos citados.

El propósito de TLAPS is verificar las pruebas de teoremas propuestas por el usurio, es decir,
que la jerarquía de las sentencias de hecho establecen la veracidad del teorema si las afirmaciones fueran ciertas,
y luego verificar que la afirmación de cada sentencia justificada es implicada por los hechos citados.

Si un teorema de TLA+ tiene una prueba con todas sus afirmaciones justificadas, entonces, como resultado
de comprobar la prueba, TLAPS verifica si el teorema es cierto.

\section{Descripción de la herramienta del lado del usuario}

Oso (pedí ayuda si es mucho!)

\section{Aspectos técnicos de la herramienta}
\section{Casos de estudio (exitosos o no) de la herramienta}
\section{Comparación con otras herramientas}
\section{Caso de estudio elegido}
\section{Conclusiones particulares}


%
% ---- Bibliography ----
%
\begin{thebibliography}{5}
%
\bibitem {pnueli}
Pnueli, Amir

The temporal logic of programs

Proceedings of the 18th IEEE Symposium on Foundation of Computer Science, 1977

\bibitem {specsys}
Lamport, Leslie

Specifying Systems: The TLA+ Language and Tools for Hardware and Software Engineers

Addison-Wesley. ISBN 0-321-14306-X, 2002




\end{thebibliography}


\clearpage
\addtocmark[2]{Author Index} % additional numbered TOC entry
\renewcommand{\indexname}{Author Index}
\printindex
\clearpage
\addtocmark[2]{Subject Index} % additional numbered TOC entry
\markboth{Subject Index}{Subject Index}
\renewcommand{\indexname}{Subject Index}
\input{subjidx.ind}
\end{document}
